\documentclass{report}
% Comment the following line to NOT allow the usage of umlauts
\usepackage[utf8]{inputenc}
\usepackage[francais]{babel}
\pagestyle{plain}


\title{Rapport de TER}
% Start the document
\begin{document}
\maketitle
\newpage
\tableofcontents
\newpage
% Create a new 1st level heading

\part{Présentation du projet}
\chapter{Introduction}
\paragraph{}
L'objectif de ce projet est la réalisation d'un jeu basé sur un modèle multi-agent. L'idée générale du projet est dans la continuité du projet de l'année dernière sur le thème. L'outil utilisé est Unity 3D, un moteur de jeu utilisé dans un grand nombre de réalisation de hautes qualités. Notre projet est utilisable sur Windows et Mac et pourrais être porté sur Android.
Ce projet consiste de réaliser un jeu que l'on peut qualifier de programmeur et de permettre, notamment, à de jeunes personnes de se familiariser avec le monde de la programmation. L'utilisateur pourra donc créer un comportement pour des robots appelé "unité" afin de remplir des objectifs du jeu.
\paragraph{}
Le projet Metabot est un projet modeste réalisé à partir du logiciel Unity 3D par un groupe d'étudiant débutant dans l'utilisation de cet outil. Malgré le peu d'expérience dans la création pure de ce genre de projet, le projet actuel est le fruit d'un travail important et d'une implication entière de toute l'équipe.
Le projet a donc pour unique prétention de communiquer notre amour du jeu vidéo et de la programmation.
\newpage
\chapter{MetaBot: Le mode par défaut}
\section{Principe}
Dans MetaBot, deux à quatre équipes se battent sur un terrain pour les ressources afin de survivre et d'éliminer les autres équipes afin d'etre la derniere ne vie. Des ressources apparaissent sur la carte et peuvent etre converti en unité ou en soin.

\newpage
\part{Réalisation du projet}
\newpage
\chapter{Partie "Moteur"}
\section{Phase de conception}
Pour réaliser ce projet, nous avons réalisé une conception basé sur les 
\section{De l'étude de l'ancien projet à la refonte totale du moteur}
\section{Réalisation}
\section{Fonctionnalités}
\section{Amélioration possible}

\newpage
\chapter{Partie "Graphisme"}

\newpage
\chapter{Partie "Interpreteur"}

\newpage
\chapter{Partie "Game Design"}

\newpage
\part{L'avenir du projet}
\chapter{Amélioration possible}

% Uncomment the following two lines if you want to have a bibliography
%\bibliographystyle{alpha}
%\bibliography{document}

\end{document}
